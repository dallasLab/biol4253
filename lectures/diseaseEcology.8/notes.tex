\documentclass[12pt]{article}
\usepackage{geometry}
\geometry{a4paper}
\usepackage[round]{natbib}
\usepackage{graphicx}
\usepackage[T1]{fontenc}
\usepackage[utf8]{inputenc}
\usepackage{textcomp}
\usepackage{gensymb}
\usepackage{hyperref}
\usepackage{amsmath}
\usepackage{amssymb}
\usepackage{authblk}
\usepackage[running]{lineno}
\usepackage{setspace}
\usepackage{rotating}

\setlength{\parindent}{0pt}


\usepackage{fancyhdr}
 
\pagestyle{fancy}
\fancyhf{}
\rhead{Tad Dallas (Biol 4253)}
\lhead{ Host-parasite interactions}

\doublespacing


\begin{document}




\subsection*{Reading:}

Kilpatrick, A. M. \& Altizer, S. (2010) Disease Ecology. Nature Education Knowledge 3(10):55 \\ \url{https://www.nature.com/scitable/knowledge/library/disease-ecology-15947677}




















\begin{center}
\noindent\hrulefill 
\end{center}



\clearpage






\subsection*{What are the effects of parasites?}

Host-parasite interactions are a type of consumer-resource interaction, where the resource is the host, and the consumer is the parasite. Here, the interaction is characterized by a gain for the parasite and a loss for the host, fundamentally differing from other symbioses such as mutualism or competition. 


Parasites are a diverse group. Pretty much every group of living thing has a parasite, including parasites (a parasite of a parasite is called a \textit{hyperparasite}). The diversity of life history strategies, transmission modes, and lifestyles make parasites unique. Parasites infect a set of hosts, referred as the \textit{host range} or the set of \textit{permissive hosts}. Meanwhile, hosts harbor any number of parasite species, which is referred as that hosts \textit{parasite species richness}. Parasites may be \textit{specialists} (infecting only a single host species) or quite \textit{generalists} (infecting a broad range of host species). 


**give examples of both of these things**



There are a couple of main ways to divide parasites into two functional groups. First, the location on the host can determine parasite grouping with \textit{endoparasites} being inside the host (e.g., cestode) and \textit{ectoparasites} living on the outside of the host (e.g., tick). Second, the size of the parasite itself can be a grouping (though this is slightly subjective), with \textit{microparasites} typically referring to those parasites that are too small to see with the naked eye, and \textit{macroparasites}, which are larger (e.g., bot flies). However one wants to divide parasites, there are a lot of parasites, and they are neat. 


Apart from the impacts of parasites on human health, agricultural crops, and livestock, parasites provide an interesting system to study and test fundamental ecological theory (some of which we've touched on previously). For instance, we can apply concepts from population dynamics to model the effect of parasites on host population dynamics. Previously, scientists demonstrated that infection of red grouse by a parasitic nematode caused the grouse population to exhibit cycles. 


**show the graph and discuss this example in more detail**



Further, we can apply the species niche concept to parasitic species to understand how the environment and host availability jointly influence parasite persistence thresholds. 
















\bigskip
\subsection*{The parasite niche}

The parasite niche is defined in a similar manner as the niche of a free-living species. That is, abiotic tolerances are important in describing the parasite niche. However, parasite lifecycle controls to what extent environmental axes are true niche axes. For instance, a parasite species which is only ever exposed to the internal environment of the host should not have a niche axis related to the temperature of the external environment, as this is unlikely to directly influence parasite persistence. For this parasite, in particular, the environment \textit{is} the host individual, such that an appropriate niche axis might relate more to some property of the host. The characterization of the parasite niche is important to understanding the geographic distribution of a parasite. That is, we can imagine situations where the environment is ideal for a parasite, but there are no \textit{permissive} host species present, so the parasite cannot persist. We can also imagine a situation where a permissive host species is present, but the environment too harsh and the parasite cannot persist in these conditions. 

Apart from using the niche concept to understand a parasite species geographic distribution, the niche concept also can allow for the estimation of \textit{host switching} potential. That is, what is the probability that a parasite can infect any given novel host, when it has not previously been found to infect this host. Host switching is important, as it can lead to \textit{spillover} events, if the host switching involves the ability to infect a human host. There may be a set of host traits which allow the prediction of host switching. That is, hosts that are more similar to known hosts of a parasite might be more likely to become infected. This makes sense in light of niche concepts, as host phylogenetic distance may represent a niche axis, along which more closely related hosts would be more likely to become infected. However, not all niche concepts can be mapped onto parasites as easy. For instance, defining persistence for a parasite which may have epidemic-like dynamics is difficult (though not impossible, as we'll discuss in the R0 section). 







\bigskip
\subsection*{Parasite life cycle complexity}

Many parasites do not simply complete their life cycle on one or more host species directly (e.g., a mosquito may feed on a subset of species directly), but instead require multiple species to complete a single generation. For instance, each segment of a tapeworm is a reproductive entity containing "eggs". These eggs are found in the feces of infected hosts. Cows which graze in these areas may consume trematodes, and larval trematodes can encyst within the muscle tissue of the host. Then humans eat the cows and get infected. This is a simple example. A more complex, and perhaps better example, is that of another trematode parasite, which has four life stages -- egg, cercariae, metacercariae, and adult -- all of which infect different host species. 


**draw this out on the board or show the slide**




Why do this, when it requires the presence of all host species to complete a single generation? That is, this if one host is removed from this, the entire parasitic life cycle falls apart. 


((( I honestly don't have a good answer for this. People say it's a response against elevated host immune function or evasion, but that seems a bit tenuous )))

























\bigskip
\subsection*{Modeling host-parasite interactions}

Despite the variety of parasite life history and transmission modes, one flexible model which can capture many fundamental aspects of host-parasite infection dynamics is the SIR model. This is a compartmental model where host individuals can be either susceptible (S), infected (I), or recovered/removed (R). Other forms of this simple model exist, including latent periods after transmission but before the host becomes infectious (E for exposed), as well as extensions for multi-host scenarios, recovery with waning immunity (i.e., recovered individuals get susceptible at some rate), and others. 


\begin{align}
\frac{dS}{dt} &=  -\beta SI\\
\frac{dI}{dt} &=  \beta SI - dI \\
\frac{dR}{dt} &=  dI
\end{align}


where $\beta$ is the transmission rate and $d$ is parasite-induced mortality or the recovery rate, depending if the $R$ class are dead or recovered with immunity. 


Model Assumptions:
1. Mass Action
2. Same susceptibility for every individual
3. All outbreaks of same disease alike


Here, the basic reproductive number ($R_0$) describes the number of secondary infections generated by a single infected individual in a wholly susceptible population. 


\begin{equation}
R_0 = \frac{\beta N}{d}
\end{equation}

**show chart of diseases by their reproductive number**

One useful thing about estimating $R_0$ is that it essentially allows the estimation of the probability that a pathogen will invade a given host population (called a \textit{pathogen invasion threshold}). That is, $R_0 >=$ 1 causes pathogen invasion, while $R_0 <$ 1 means the pathogen does not invade. Also, we can look at each component of $R_0$, and get an idea of how to control a disease. That is, $R_0$ is made up of three terms:

\begin{itemize}
  \item N the total number of individuals in a wholly susceptible population
  \item $\beta$ the transmission rate
  \item $d$ parasite-induced mortality (i.e., the recovery rate)
\end{itemize}

If $d$ is big, the pathogen cannot invade. 
If $N$ or $\beta$ are big, the pathogen can invade. 



Loosely speaking, the outcome of an initial pathogen infection event in the SIR model can have 3 possible outcomes. First, the $R_0$ is below 1, and the infected individual moves to the recovered/removed class without causing any further infection. Second, the pathogen invades, infects a bunch of people, and then leaves the population (an epidemic pathogen). Third, the pathogen invades, and maintains sustained low levels of infection in the population (an endemic pathogen). 


\paragraph*{Vaccination}

What happens if we can pre-emptively treat some fraction of the population? This would lower the total $N$, thereby reducing $R_0$ for that population. One way we can do this is by stopping infection for a group of individuals. It is possible to vaccinate enough of the population that the $R_0$ is reduced below 1, and the entire population is not susceptible to pathogen invasion. This is called \textit{herd immunity}, and the threshold fraction of the population required to vaccinate to acheive herd immunity is $1 - \frac{1}{R_0}$. 




\bigskip
\subsection*{Evolution of virulence and the Red Queen}

But $R_0$ isn't necessarily a static quantity, but can potentially change through time. This occurs for a number of reasons. One of the reasons is that transmission parameters can change. Hosts and parasites are in a constant arms race, where hosts evolve resistance mechanisms to the pathogen, and the pathogen evolves ways to get around these host defenses. This process of host-parasite co-evolution is referred to as \textit{Red Queen dynamics}.

**examples/evidence of Red Queen**











\bigskip
\subsection*{Community composition and disease risk}

Host species differ in their susceptibility (ability to become infected) and suitability (ability to transmit parasites). On top of this, most parasite species infect more than one host species. Therefore, the diversity of host species present in a given site may influence the resulting parasite infection dynamics. These relationships are often referred to as \textit{diversity-disease} relationships. On one hand, increased host diversity may reduce overall parasite burden or abundance through the addition of host species which are less suitable, lowering overall transmission risk or parasite abundance. This is called a \textit{dilution effect}. On the other hand, if species in more diverse communities are \textbf{more} suitable, this would likely increase overall transmission risk or parasite abundance, leading to an \textit{amplification effect}. 


**evidence and examples; classic ostfeld keesing stuff, and then johnson work showing the importance of the community composition**



































\end{document}
