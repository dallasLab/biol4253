\documentclass[12pt]{article}
\usepackage{geometry}
\geometry{a4paper}
\usepackage[round]{natbib}
\usepackage{graphicx}
\usepackage[T1]{fontenc}
\usepackage[utf8]{inputenc}
\usepackage{textcomp}
\usepackage{gensymb}
\usepackage{hyperref}
\usepackage{amsmath}
\usepackage{amssymb}
\usepackage{authblk}
\usepackage[running]{lineno}
\usepackage{setspace}
\usepackage{rotating}

\setlength{\parindent}{0pt}


\usepackage{fancyhdr}
 
\pagestyle{fancy}
\fancyhf{}
\rhead{Tad Dallas (Biol 4253)}
\lhead{ Trophic interactions}

\doublespacing


\begin{document}



\subsection*{Reading:}

Gotelli, Nicholas J. A primer of ecology. Sunderland, MA: Sinauer Associates, 2001. Chapter 6.\\

\bigskip

Hui, D. (2012) Food Web: Concept and Applications. Nature Education Knowledge 3(12):6 \url{https://www.nature.com/scitable/knowledge/library/food-web-concept-and-applications-84077181}








\begin{center}
\noindent\hrulefill 
\end{center}



\clearpage

\subsection*{What controls consumer-resource interactions and dynamics?}

This week, we'll look at consumer-resource interactions a bit more explicitly. These were mentioned briefly along with food webs in the community lecture, but now we'll go into more detail. Consumer-resource interactions are a broad class of interactions that include predator-prey, plant-herbivore, and host-parasite interactions. The availability of resource fundamentally constrains the number of predators that can exist. In the logistic model, we assumed that population dynamics were constrained by some carrying capacity $K$, which could be driven by resources, but we did not consider resources explicitly. The availability of resources strongly influences predator populations. 



**show the data on lynx and snowshoe hare**





Let's see what happens to single species population dynamics when we do consider the resource population explicitly. We'll start by considering the \textit{Lotka-Volterra model}, which tracks consumer ($C$) and resource ($R$) populations through time. 


The model makes several simplifying assumptions: 1) the prey population will grow exponentially when the predator is absent; 2) the predator population will starve in the absence of the prey population (as opposed to switching to another type of prey); 3) predators can consume infinite quantities of prey; and 4) there is no environmental complexity (in other words, both populations are moving randomly through a homogeneous environment).


So let's first consider the case of the consumer in the absence of the resource. 

\begin{equation}
\frac{dC}{dt} = -qC
\end{equation}

where $q$ is the consumer ($C$) mortality rate. Without resources, the consumer will decline exponentially to extinction. Now we'll add resources into the consumer equation.


\begin{equation}
\frac{dC}{dt} = caCR - qC
\end{equation}


Here, the term $caCR$ corresponds to the attack rate ($a$) times the conversion of food into offspring ($c$) times the abundance of both consumer $C$ and resource $R$. This suggests that consumer population growth is fundamentally and closely linked to the abundance of resource $R$. The population dynamics of the resource population are similar to those of consumer, except there is no assumed background mortality rate. Instead, the population $R$ grows exponentially at rate $r$, but the population is reduced by the effect of the consumer ($aCR$). That is, 


\begin{equation}
\frac{dR}{dt} = rR - aCR
\end{equation}


Depending on the parameterization of the model (e.g., attack rate $a$, conversion rate $c$, growth rate of the resource $r$, etc.), this model can display a range of dynamic behaviors. One interesting result from this simple model is the existence of sustained oscillatory behavior. The behavior is caused by the inherent feedback between consumer and resource. That is, the consumer and resource oscillate together through time, with the consumer lagged forward in time relative to the resource dynamics. This suggests that resource populations drive consumer dynamics, where resources are allowed to increase at relatively low consumer abundance, but high resource abundance increases consumer populations, which serves to drive down resource populations. We will explore the sensitivity of these fluctuations to model parameterization in the coding exercise. 


This behavior has been observed in real consumer-resource systems as well, supporting the theoretical expectation derived from the Lotka-Volterra model. 

In a famous experiment in 1958, Huffaker examined the consumer-resource interaction between two mite species, finding sustained oscillations. 

**show data on Huffaker's mites**


However, it is important to note that Huffaker had to build in \textbf{a lot} of things to help the resource population persist. In essence, this would correspond to artificially lowering the attack rate of the predator. For instance, the experiment consisted of oranges (food resource to the resource species) and orange balls interspersed (serving to increase habitat complexity), and these were spread far apart (creating a metapopulation, essentially). This attempted to provide the resource species a bit of refuge from the predator, who would have otherwise just consumed the resource to extinction. So while much can be learned from the Lotka-Volterra model, it is a simple representation of consumer-resource dynamics which assumes a self-limiting consumer population, homogenous mixing, and a constant attack rate. 















\bigskip
\subsection*{Food webs and their structure}

The Lotka-Volterra model examines the interactions between a single consumer and resource species. What if we scale this up to a set of interacting consumer and resource species? While models exist to describe these interactions (especially in the case of a single resource population with many consumers), we will discuss this situation in more conceptual terms. The feeding interactions across \textit{trophic levels} form the \textit{food web}, which describes all the trophic interactions among species in a given location. 



**show an example food web**




Here, the food web is depicted as a graph (a network), where nodes of the network are species and links are directional feeding associations describing the flow of energy from one species to another via a "feeding" interaction. Food webs are typically broken down into trophic levels, forming a \textit{trophic pyramid} where each level of the pyramid corresponds to a set of species which occupy the same trophic level. The base of the food pyramid (or food web) is most commonly composed of \textit{autotrophic} species -- also called primary producers -- which are photosynthetic organisms. The next trophic levels consist of \textit{heterotrophic} species. Theh immediate next level are often the small herbivores which consume the autotrophs, and after them are the \textit{primary predators} who consume the herbivores. After this are \textit{secondary predators} and so on. 

**show differences between food web and food pyramid**









\bigskip
\subsection*{What controls food webs?}

But why do we depict food webs as pyramids in the first place? It's a natural way to showcase the flow of energy to higher trophic levels, but it also often corresponds to the reduction in species richness at each trophic level. That is, there are often many autotrophic species, while there are typically very few secondary predator species. In fact, the total number of trophic levels is fundamentally limited by the flow of energy. 

To explore this more, we'll consider how each trophic level in the pyramid differs in terms of overall abundance and biomass. In terms of abundance, the pyramid shape is maintained, with lower trophic levels typically being overall more abundant. This relationship is even more pronounced when we consider biomass, as autotrophic species generally have quite high biomass. So why is this pyramid shape maintained?

Every trophic interaction represents a flow of energy from one level to the next. But these interactions aren't without waste. That is, the ability of the consumer to convert the resource into energy is not 100\%, and consumers need to consume many resource items to create one new consumer (as we saw in the Lotka-Volterra model). This trophic difference in ability to capture energy is sometimes referred to as the \textit{pyramid of productivity}, which posits that energy transfer between trophic levels to create consumer biomass results in only 10\% of the consumed energy used to create new consumer biomass. This suggests that each trophic level will be 10\% of the size of the previous level. This is an oversimplification, but it's also a good piece of conceptual theory. This also helps explain why food webs tend to have fewer than five trophic levels, as starting with an autotrophic biomass pool of 100,000 units will result in 10 units of predator biomass in the fifth trophic level.










\bigskip
\subsection*{The influence of the environment on food webs}

Fluctuating environments should have smaller food webs. 






















\bigskip
\subsection*{Trophic cascades}

Combining concepts of the Lotka-Volterra model to the entire food web scale, what would happen if one trophic level experienced a perturbation? For instance, hunting pressure increased and reduced the abundance of deer, which serve as herbivores? We might expect, based on Lotka-Volterra assumptions, that the autotrophs would increase in abundance, being freed slightly from the influence of consumption. This would correspond to a \textit{top-down trophic cascade}. We can also consider what the reduction in herbivore abundance would mean for higher trophic levels, as we might expect a reduction in abundance of higher tropic levels. This is sometimes called a \textit{bottom-up trophic cascade}, but it's hardly a cascade in the true sense of the term. That is, a true trophic cascade differentially affects trophic levels. A classic example is the relationship between otters, sea urchins, and kelp forests. Otters are top consumer, eating the sea urchins that consume the kelp. If we reduce otter abundances in this situation, it would cause an increase in sea urchin abundance, as they become freed from predation. This, in turn, reduces kelp abundance, creating a situation where sea otters (the highest trophic level) fluctuate positively with kelp abundance (increases in otters cause increses in kelp). 


















\bigskip
\subsection*{}




\end{document}

