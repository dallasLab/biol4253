\documentclass[12pt]{article}
\usepackage{geometry}
\geometry{a4paper}
\usepackage[round]{natbib}
\usepackage{graphicx}
\usepackage[T1]{fontenc}
\usepackage[utf8]{inputenc}
\usepackage{textcomp}
\usepackage{gensymb}
\usepackage{amsmath}
\usepackage{amssymb}
\usepackage{authblk}
\usepackage[running]{lineno}
\usepackage{setspace}
\usepackage{rotating}
\usepackage{hyperref}

\setlength{\parindent}{0pt}


\usepackage{fancyhdr}
 
\pagestyle{fancy}
\fancyhf{}
\rhead{Tad Dallas (Biol 4253)}
\lhead{Macroecology}

\doublespacing




\begin{document}










\subsection*{Reading:}

McGill, Brian J. "The what, how and why of doing macroecology." Global Ecology and Biogeography 28.1 (2019): 6-17. \\ \url{https://onlinelibrary.wiley.com/doi/pdf/10.1111/geb.12855} \\

\bigskip

Shade, A., Dunn, R. R., Blowes, S. A., Keil, P., Bohannan, B. J., Herrmann, M., ... \& Chase, J. (2018). Macroecology to unite all life, large and small. Trends in Ecology \& Evolution. \\ \url{https://www.indiana.edu/~microbes/publications/Shade_etal_2018_InPress.pdf} \\











\begin{center}
\noindent\hrulefill 
\end{center}



\clearpage



\subsection*{How do we scale small scale processes to global scales?}

Introduce the study of macroecology, with all it's bullshit. 

















\bigskip
\subsection*{Latitudinal scaling}


Diversity, range size (Rappoport), etc. 












\bigskip
\subsection*{Allometric scaling}

METE, etc.
















\bigskip
\subsection*{Species abundance distributions}

SAD, abundant-centre, 























\bigskip
\subsection*{Conclusions}
















\end{document}
